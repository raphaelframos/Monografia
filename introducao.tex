\chapter{Introdução}

A fala é a principal forma de comunicação dos seres humanos, desde o início dos computadores, a busca por computadores mais inteligentes, levam cientistas ao estudo de sistemas de \textit {reconhecimento automático de voz}, visando uma comunicação natural entre o homem e a máquina. \cite{RavPtBr}
Os sistemas de reconhecimento automático de voz \textit{(RAV)} evoluiram considerávelmente com o passar dos anos, e 
sua aplicação se encontra em diversas áreas, como: sistemas para atendimento automático, ditado, interfaces para computadores pessoais, controle de equipamentos, robôs domésticos, indústrias totalmente à base de robôs inteligentes, etc. \cite{RavPtBr} Mas mesmo com toda evolução do hardware dos computadores e otimização dos algoritmos e métodos, os sistemas \textit{(RAV)} estão longe de compreender um discurso sobre qualquer assunto, falado de forma natural, por qualquer pessoa, em qualquer ambiente.\cite{RavIsoladas} 
   

\section{Objetivos}
O objetivo geral deste trabalho é desenvolver um jogo interativo guiado por comandos voz ditados pelo usuário, o jogo é baseado em um clássico do mundo dos games, pacman, onde o objetivo do personagem principal é comer todas as pastilhas, e não ser devorado pelos 4 fanstamas que o perseguem por um labirinto. A interação é feita usando comandos de fala pré-definidos em sua gramática, que são: DIREITA, ESQUERDA, SUBIR, DESCER. Além de ser guiado por esses comandos, o jogo também reconhece determinadas palavras que podem caracterizar o humor do usuário, como: BURRO, DROGA, MERDA, pronunciadas essas ofensas, o usuário recebe uma penalidade, até perder a partida. 

\section{Visão Geral do Trabalho}
