\chapter{Introdução}

A fala é a principal forma de comunicação dos seres humanos, desde o início dos computadores, a busca por computadores mais inteligentes, levam cientistas ao estudo de Sistemas de \textit {Reconhecimento Automático de Voz} visando uma comunicação natural entre o homem e a máquina, interação vista apenas em filmes de ficção científica \cite{RavPtBrCarlos}. Esses sistemas também são chamados de \textit{(RAV)}. 
Para esses estudos virarem realidade, os computadores terão de possuir total entendimento da fala humana, capacidades como: falar, ouvir, ler, escrever, alem do reconhecimento de pessoas pela voz, devem ser estabelecidas. Essas capacidades são os objetivos dos sistemas de \textit{RAV}, permitindo que o computador “entenda” o que está sendo dito \cite{RavCorporaCarlos}.
Os sistemas de \textit{RAV} evoluiram considerávelmente com o passar dos anos, e sua aplicação se encontra em diversas áreas, como: sistemas para atendimento automático, ditado, interfaces para computadores pessoais, controle de equipamentos, robôs domésticos, indústrias totalmente à base de robôs inteligentes, segurança etc \cite{RavPtBrCarlos}. Mas mesmo com toda evolução do hardware dos computadores e otimização dos algoritmos e métodos, os sistemas \textit{RAV} estão longe de compreender um discurso sobre qualquer assunto, falado de forma natural, por qualquer pessoa, em qualquer ambiente \cite{RavIsolAnderson}. 

Com a tendência da melhora dos processadores, memórias e placas de vídeo, esta linha de processamento de voz está em bastante evidência. Os jogos de computadores, são uma área que também está acompanhando essa evolução, e também demandam reconhecimento de voz para controle de comandos. Jogos de computadores, se tornaram cada vez mais parecidos com a realidade em gráficos e na interatividade, a tendencia sugere que os famosos joysticks poderão ser aposentados em pouco tempo.

    O primeiro jogo, foi desenvolvido em 30 de julho de 1961, por Steve Russel, que não tinha objetivos comerciais, apenas acadêmicos. O principal objetivo de Steve Russel era poder mostrar todo o poder de processamento do computador DEC PDP-1, para isso foi criado o SpaceWar. Inicialmente a ideia de Russel era fazer um filme interativo, mas acabou se tornando o pai dos jogos eletrônicos \cite{HistJogosHenrique}. Milhares de jogos foram desenvolvidos nas décadas seguintes, passando por tetrix do russo Alexey Pajitnov, super Mário que foi o jogo mais vendido da época, até chegar nos games atuais, que são quase um espelho da realidade. Videogames com as mais modernas tecnologias vem sendo lançados ultimamente, um exemplo é o Xbox 360, fabricado pela Microsoft Corporation, que surpreendeu ao fazer o joystick que possui um sistema inteligente de profundidade de seus botões traseiros, similares a um gatilho. Com isso, os comandos são interpretados de acordo com a intensidade em que estes são pressionados. Em um game de corrida, por exemplo, faz uma enorme diferença na hora de acelerar mais suavemente com o seu carro, ou simplesmente “afundar” o pé no acelerador \cite{XBoxTechT}. Mas a grande revolução ainda estava por vir, em novembro de 2010, a Microsoft lançou o kinect, um sensor de movimento que veio para revolucionar o mundo dos games, promovendo uma integração total com o jogador, e acabando com a mística de que jogar videogame é sinal de sedentarimo \cite{KinectTechT}.

\section{Objetivos}
O objetivo geral deste trabalho é desenvolver um jogo interativo guiado por comandos de voz ditados pelo usuário, o método abordado será testado em um clássico do mundo dos games, pacman, onde o objetivo do personagem principal é comer todas as pastilhas, e não ser devorado pelos 4 fanstamas que o perseguem por um labirinto. A interação é feita usando comandos de fala pré-definidos em sua gramática, que são: DIREITA, ESQUERDA, SUBIR, DESCER. Além de ser guiado por esses comandos, o jogo também reconhece determinadas palavras que podem caracterizar o humor do usuário, como: BURRO, DROGA, MERDA, pronunciadas essas ofensas, o usuário recebe uma penalidade, até perder a partida. 

\section{Motivações}
A maior motivação seria o aumento de desempenho individual, pois sendo o meio de comunicação mais natural para o ser humano, os comandos por voz seriam mais rápidos que por joystick, permitindo utilizar as mãos para fazer outras coisas em quanto estivesse jogando, além das diversas aplicações que são cada vez maiores nas atividades humanas.

\section{Aplicações do reconhecimento automático de fala}
Segundo \citeonline{AvaliaTecJose} os sistemas com reconhecimento de voz podem ser aplicados em qualquer atividade que demande interação homem-máquina, e nas mais diversas áreas. Há mais de uma década ele já mostrava a importância do uso de voz em diversas aplicações, algumas dessas áreas são:

\begin{itemize}
\item Sistemas de controle e comando: Estes sistemas utilizam a fala para realizar
determinadas funções;
\item Sistemas de telefonia: O usuário pode utilizar a voz para fazer uma chamada, ao
invés de discar o número;
\item Sistemas de transcrição: Textos falados pelo usuário podem ser transcritos
automaticamente por estes sistemas;
\item Acesso à informação: O usuário recebe algum tipo de informação, que se encontra armazenada em um banco de dados. Exemplo: notícias, previsão do tempo, hora certa, etc.
\item Centrais de atendimento ao cliente: Uma atendente virtual pode ser utilizada a fim
de realizar o atendimento ao cliente;
\item Operações bancárias: O usuário efetua operações bancárias, como informações do seu saldo, trasnferências de dinheiro.
\item Preenchimento de formulários: O usuário entra com os dados via fala.
\item Robótica: Robôs podem se comunicar pela fala com seus donos.

\end{itemize}

Como essas áreas são muito importantes para os seres humanos, muitos produtos utilizam a fala pra realizações de ações. Como exemplo temos:

\citeonline{RavCadeiraAedb} desenvolve uma aplicação de reconhecimento de voz para aplicações em cadeira de rodas, onde o cadeirante se movimenta através de comandos de voz, como facilitador da implementação, foi utilizado o software IBM Via Voice, que segundo \citeonline{ImplementServVirtuaisDamasceno} obteve um melhor desempenho e aplicabilidade quando comparado a outros softwares, considerando a língua falada, a robustez do reconhecimento e a interface de trabalho com outros programas devido à aplicação deste desenvolvimento ser no Brasil.

Já em \citeonline{AcionamentRoboLegoFabio}, para efetuar o reconhecimento de voz foi utilizado redes neurais artificiais, também chamadas de \textit{(RNA)}. Usando como base RNA foi criado uma rede para identificar comandos básicos de voz, e assim, efetuar o acionamento de um robô móvel. Outra característica importante no projeto é o identificador neural, que foi desenvolvido como dependente do locutor, onde um sistema é desenvolvido com base nas características vocais de um locutor. Para novos locutores seria necessário um novo treinamento da rede com as características vocais dos novos locutores.

\section{Metodologia}
Para o propósito deste trabalho se estabeleceu estudos de técnicas de reconhecimento de voz para comandos, técnicas de análise de fala, extração de atributos que serão alimentados como treinamento para um modelo de grafos de cadeias de markov oculto, implementação dos diferentes modelos, acoplação no jogo clássico pacman e verificação dos resultados. A implementação será desenvolvida para dispositivos móveis com sistema operacional android, que utiliza como linguagem de programação a linguagem Java.

\section{Visão Geral do Trabalho}
Neste trabalhou buscou-se desenvolver um sistema \textit{RAV} com baixa taxa de erros, um dicionário pequeno de palavras e reconhecimento de palavras isoladas, que são a melhor forma de introdução nos estudos de sistemas com reconhecimento de fala, possibilitando estudos futuros em aplicações com reconhecimento de palavras contínuas e vocabulários grandes. O sistema desenvolvido é baseado nos modelos ocultos de Markov \textit{(HMM)}. 

Este trabalho está dividido em %colocar a quantidade de cap. 
capítulos, que são descritos a seguir:

O Capítulo 2 tem como objetivo fazer as referências teóricas sobre o processamento do sinal de fala, sistemas \textit{RAV}, como suas características, histórico e reconhecimento de padrões.









